%\PassOptionsToPackage{gray}{xcolor}
\usepackage[utf8]{inputenc}
%\usepackage[dvipsnames,table]{xcolor}
%\usepackage{array}
\usepackage{tikz}
\usepackage{environ}
\usepackage{colortbl}

\usepackage[scaled]{helvet}
\renewcommand\familydefault{\sfdefault}
%\usepackage[scaled]{FiraMono}
\usepackage{inconsolata}
\usepackage[T1]{fontenc}

\usetikzlibrary{positioning, arrows.meta}
\usetikzlibrary{shapes.arrows}
\usetikzlibrary {shapes.callouts}
\usetikzlibrary{shapes.misc}

\usetikzlibrary{decorations.pathreplacing}
\usetikzlibrary{calc, fit, positioning}
\usetikzlibrary{calligraphy}
\usetikzlibrary{shapes.geometric}
\usetikzlibrary {shapes.symbols}
\usetikzlibrary{decorations.markings}

% packet's colors -- was in tikzpacket.tex
% from parula color scheme
\definecolor{parDarkBlue}{HTML}{3d26a8}
\definecolor{parBlue}{HTML}{4757f7}
\definecolor{parLightBlue}{HTML}{2796eb}
\definecolor{parGreenBlue}{HTML}{18bfb5}
\definecolor{parGreen}{HTML}{80cb58}
\definecolor{parOrange}{HTML}{fdbd3c}
\definecolor{parYellow}{HTML}{f9fa14}

%\colorlet{hdrBgSix}{parLightBlue!20}
%\colorlet{hdrBgSixLight}{white} %{parLightBlue!10}
%\colorlet{hdrBgFour}{parGreen!20}
%\colorlet{hdrBgFourLight}{white} %{parGreen!10}
%\colorlet{hdrBgSRH}{parOrange!20}
%\colorlet{hdrBgSRHLight}{white} %{parOrange!20}
%\colorlet{hdrBgUDP}{parGreenBlue!20}
\colorlet{hdrBgPayload}{gray}
\colorlet{hdrTxtSix}{black}
\colorlet{hdrTxtSixLight}{black}
\colorlet{hdrTxtFour}{black}
\colorlet{hdrTxtFourLight}{black}
\colorlet{hdrTxtSRH}{black}
\colorlet{hdrTxtSRHLight}{black}
\colorlet{hdrTxtUDP}{black}
\colorlet{hdrTxtPayload}{white}

% derived using LAB color scheme and ppt colors
\definecolor{labOrange}{HTML}{fed147}
\definecolor{labGreen}{HTML}{8ada64}
\definecolor{labBlue}{HTML}{00b9e8}
\definecolor{labRed}{HTML}{fe4a35}

\colorlet{hdrBgSix}{labBlue!20}
\colorlet{hdrBgSixLight}{white} %{labBlue!10}
\colorlet{hdrBgFour}{labGreen!20}
\colorlet{hdrBgFourLight}{white} %{labGreen!10}
\colorlet{hdrBgSRH}{labOrange!20}
\colorlet{hdrBgSRHLight}{white} %{labOrange!10}
\colorlet{hdrBgUDP}{labRed!20}

\definecolor{col_algo0}{HTML}{4EA72E} % green
%\colorlet{col_algo0}{gray}
\definecolor{col_algo128}{HTML}{00ADDC} % blue
%\colorlet{col_algo128}{parBlue}
\definecolor{col_algo129}{HTML}{00ADDC} % blue
%\colorlet{col_algo129}{parGreen}
\colorlet{col_TOFIX}{pink}

\definecolor{color0}{HTML}{999999}
\definecolor{color1}{HTML}{E69F00}
\definecolor{color2}{HTML}{56B4E9}
\definecolor{color3}{HTML}{009E73}
\definecolor{color4}{HTML}{F0E442}
\definecolor{color5}{HTML}{0072B2}
\definecolor{color6}{HTML}{D55E00}
\definecolor{color7}{HTML}{CC79A7}

% colors supposedly having same gray value as 50% gray
\definecolor{VIOLET}{HTML}{A66DDF}
\definecolor{MAGENTA}{HTML}{DD59DD}
\definecolor{AZURE}{HTML}{3F8BD7}
\definecolor{CYAN}{HTML}{219C9C}

\definecolor{VRFGREEN}{HTML}{4EA72E} % green
%\colorlet{VRFGREEN}{parGreen}
\definecolor{VRFBLUE}{HTML}{00ADDC} % blue
%\colorlet{VRFBLUE}{parBlue}
\definecolor{EVI100}{HTML}{DA9202} % orange/brown
%\colorlet{EVI100}{parOrange}

%\definecolor{EXPLOSION}{HTML}{FFA500}
%\definecolor{EXPLOSIONTEXT}{HTML}{FFFFFF}
\colorlet{EXPLOSION}{orange}
\colorlet{EXPLOSIONTEXT}{white}

%\definecolor{IGPMETRIC}{HTML}{00FF00}
\colorlet{IGPMETRIC}{black}
%\definecolor{DELAYMETRIC}{HTML}{0000FF}
\colorlet{DELAYMETRIC}{black}

\colorlet{GRAYEDOUTBORDER}{gray}
\colorlet{GRAYEDOUTFILL}{gray}
\colorlet{GRAYEDOUTTEXT}{white}

\colorlet{PKTANNOT}{gray} % packet annotations

\colorlet{EBGPV4}{CYAN}%{purple}
\colorlet{IBGPV4}{AZURE}%{purple}
\colorlet{EBGPV6}{CYAN}%{brown}
\colorlet{IBGPV6}{AZURE}%{orange}

\colorlet{REDISTTAG}{purple} % 0050 figure 9

\colorlet{AFFINITYRED}{purple} % 0040

\colorlet{VLAN1}{purple}
\colorlet{VLAN2}{brown}
\colorlet{VLAN3}{orange}

\colorlet{BUMTRAFFIC}{VRFGREEN}

\colorlet{BLOCKED}{purple}

% packet border
\colorlet{IPV4}{purple}
\colorlet{IPV6}{brown}

% GIB/LIB diagram
\colorlet{GIBFILL}{black!10}
\colorlet{LIBFILL}{black!20}

% iACL and eACL
\colorlet{IACL}{purple}
\colorlet{EACL}{brown}

\colorlet{GENERICDOMAIN}{gray}
\colorlet{DOMZERO}{gray}%{brown}
\colorlet{DOMONE}{gray}%{orange}
\colorlet{DOMTWO}{gray}%{purple}


\ifdefined\tikzpacketIsLoaded
  % contentr of file has already been included, do nothing
\else
  % Define the flag to prevent future inclusions
  \def\tikzpacketIsLoaded{}



\newsavebox{\tableboxa}

\providecommand{\smallpktwidth}{2.5cm}
\providecommand{\stdpktwidth}{5.1cm}
\providecommand{\largepktwidth}{5.9cm}
\providecommand{\xlargepktwidth}{6.6cm}

% color definitions moved to colors.tex

\providecommand{\cbgPayload}{\cellcolor{hdrBgPayload}\color{hdrTxtPayload}}
\providecommand{\cbgFour}{\cellcolor{hdrBgFour}\color{hdrTxtFour}}
\providecommand{\cbgSix}{\cellcolor{hdrBgSix}\color{hdrTxtSix}}
\providecommand{\cbgSixLight}{\cellcolor{hdrBgSixLight}\color{hdrTxtSixLight}}
\providecommand{\cbgSRH}{\cellcolor{hdrBgSRH}\color{hdrTxtSRH}}
\providecommand{\cbgSRHLight}{\cellcolor{hdrBgSRHLight}\color{hdrTxtSRHLight}}
\providecommand{\cbgUDP}{\cellcolor{hdrBgUDP}\color{hdrTxtUDP}}

\NewEnviron{srv6packet}[1][\stdpktwidth]{%
    \savebox{\tableboxa}{%
    \setlength\extrarowheight{5pt}%
    \begin{tabular}{m{1.3cm} m{#1}}
        \BODY%
        \multicolumn{2}{c}{\cbgPayload Payload}\\[3pt]
    \end{tabular}}%
    \begin{tikzpicture}
        \begin{scope}
            \clip[rounded corners=1ex] (0,-\dp\tableboxa) -- (\wd\tableboxa,-\dp\tableboxa) -- (\wd\tableboxa,\ht\tableboxa) -- (0,\ht\tableboxa) -- cycle;
            \node at (0,-\dp\tableboxa) [anchor=south west,inner sep=0pt]{\usebox{\tableboxa}};
        \end{scope}
        \draw[rounded corners=1ex] (0,-\dp\tableboxa) -- (\wd\tableboxa,-\dp\tableboxa) -- (\wd\tableboxa,\ht\tableboxa) -- (0,\ht\tableboxa) -- cycle;
    \end{tikzpicture}
}

\NewEnviron{srv6packet2}{%
    \savebox{\tableboxa}{%
    \setlength\extrarowheight{5pt}%
    \begin{tabular}{m{1.3cm} m{\stdpktwidth}}
        \BODY%
        \multicolumn{2}{c}{\cbgPayload Payload}\\[3pt]
    \end{tabular}}%
    \begin{tikzpicture}
        \begin{scope}
            \clip[rounded corners=1ex] (0,-\dp\tableboxa) -- (\wd\tableboxa,-\dp\tableboxa) -- (\wd\tableboxa,\ht\tableboxa) -- (0,\ht\tableboxa) -- cycle;
            \node at (0,-\dp\tableboxa) [anchor=south west,inner sep=0pt]{\usebox{\tableboxa}};
        \end{scope}
        \draw[rounded corners=1ex] (0,-\dp\tableboxa) -- (\wd\tableboxa,-\dp\tableboxa) -- (\wd\tableboxa,\ht\tableboxa) -- (0,\ht\tableboxa) -- cycle;
    \end{tikzpicture}
}

\newcommand{\ipfourhdr}[2]{\cbgFour IPv4 Hdr & (\texttt{#1}, \texttt{#2})\\[3pt]\hline}
\newcommand{\ipsixhdr}[2]{\cbgSix IPv6 Hdr & (\texttt{#1}, \texttt{#2})\\[3pt]\hline}
%\newcommand{\srhdr}[2]{\cbgSRH SR Hdr & (#1)\hspace{1cm}SL~=~#2\\[3pt]\hline}
\newcommand{\srhdr}[2]{\cbgSRH SR Hdr & %
    \begin{tikzpicture}%
        \node[align=left, inner xsep=0] (xx) {\vspace{3pt}(#1)\\SL~=~#2};%
    \end{tikzpicture}\\[3pt]\hline}

\newcommand{\smallipsixpkt}[3][\smallpktwidth]{%
    \begin{srv6packet}[#1]%
        \ipsixhdr{#2}{#3}%
    \end{srv6packet}
}

\newcommand{\ipsixpkt}[3][\stdpktwidth]{%
    \begin{srv6packet}[#1]%
        \ipsixhdr{#2}{#3}%
    \end{srv6packet}
}

\newcommand{\largeipsixpkt}[3][\largepktwidth]{%
    \begin{srv6packet}[#1]%
        \ipsixhdr{#2}{#3}%
    \end{srv6packet}
}

\newcommand{\smallipfourpkt}[3][\smallpktwidth]{%
    \begin{srv6packet}[#1]%
        \ipfourhdr{#2}{#3}%
    \end{srv6packet}
}

\newcommand{\ipfourpkt}[3][\stdpktwidth]{%
    \begin{srv6packet}[#1]%
        \ipfourhdr{#2}{#3}%
    \end{srv6packet}
}

\newcommand{\ipsrhpkt}[5][\stdpktwidth]{%
    \begin{srv6packet}[#1]%
        \ipsixhdr{#2}{#3}%
        \srhdr{#4}{#5}%
    \end{srv6packet}
}

\newcommand{\ipfourinsixpkt}[5][\stdpktwidth]{%
    \begin{srv6packet}[#1]%
        \ipsixhdr{#2}{#3}%
        \ipfourhdr{#4}{#5}%
    \end{srv6packet}
}

\newcommand{\largeipfourinsixpkt}[5][\largepktwidth]{%
    \begin{srv6packet}[#1]%
        \ipsixhdr{#2}{#3}%
        \ipfourhdr{#4}{#5}%
    \end{srv6packet}
}

\newcommand{\xlargeipfourinsixpkt}[5][\xlargepktwidth]{%
    \begin{srv6packet}[#1]%
        \ipsixhdr{#2}{#3}%
        \ipfourhdr{#4}{#5}%
    \end{srv6packet}
}

\newcommand{\ipsixinsixpkt}[5][\stdpktwidth]{%
    \begin{srv6packet}[#1]%
        \ipsixhdr{#2}{#3}%
        \ipsixhdr{#4}{#5}%
    \end{srv6packet}
}

\newcommand{\largeipsixinsixpkt}[5][\largepktwidth]{%
    \begin{srv6packet}[#1]%
        \ipsixhdr{#2}{#3}%
        \ipsixhdr{#4}{#5}%
    \end{srv6packet}
}


\fi

\ifdefined\tikznetworkIsLoaded
  % contentr of file has already been included, do nothing
\else
  % Define the flag to prevent future inclusions
  \def\tikznetworkIsLoaded{}

\tikzstyle{basenode}=[circle, draw, outer sep=1pt, fill=white]

%\tikzstyle{router}=[basenode, minimum size=.35cm]
\tikzstyle{router}=[basenode, minimum size=8mm]

\tikzstyle{bigrtr}=[router, very thick, scale=3] % at scale 3, default thin (0.4pt) becomes very thick (1.2pt)
\tikzstyle{bigrtr_external}=[bigrtr, draw=gray, text=gray]
\tikzstyle{bigrtr_transit}=[bigrtr, thin, inner sep=0]

\tikzstyle{connec}=[color=black]
\tikzstyle{srpath}=[very thick, ->, >=latex]
\tikzstyle{srpath_alt}=[very thick, dashed]

% very thick = 1.2pt
% colors in colors.tex
%\colorlet{col_algo0}{gray}
%\colorlet{col_algo128}{parBlue}
%\colorlet{col_algo129}{parGreen}
%\colorlet{col_TOFIX}{pink}

\tikzstyle{srpath0}=[very thick, col_algo0, >=latex]
\tikzstyle{srpath0_alt}=[srpath0, dashed]
%\tikzstyle{srpath0_tilfa}=[very thick, double, black, >=latex]
\tikzstyle{srpath0_tilfa}=[srpath0, dotted]
\tikzstyle{srpath0_tilfa_alt}=[srpath0_tilfa, dashed, col_TOFIX]
\tikzset{squarepost/.style={
    decoration={
        markings,
        mark=between positions 0.1 and 0.9 step 10mm with {
            \node[shape=rectangle, minimum size=3.6pt, fill=col_algo128, inner sep=0.5pt, transform shape] {};
        }
    },
    postaction={decorate}
}}
\tikzset{squaremid/.style={
    decoration={
        markings,
        mark=at position 0.5 with {
            \node[shape=rectangle, minimum size=3.6pt, fill=col_algo128, inner sep=0.5pt, transform shape] {};
        }
    },
    postaction={decorate}
}}
\tikzstyle{srpath128}=[very thick, col_algo128, >=latex]
\tikzstyle{srpath128_tilfa}=[srpath128, dotted]
\tikzstyle{srpath128_alt}=[srpath128, dashed, col_TOFIX]
\tikzset{circlepost/.style={
    decoration={
        markings,
        mark=between positions 2*3.6pt and -2*3.6pt step 10mm with {
            \node[shape=circle, minimum size=3.6pt, fill=col_algo129, inner sep=0.5pt, transform shape] {};
        }
    },
    postaction={decorate}
}}
\tikzset{circlemid/.style={
    decoration={
        markings,
        mark=at position 0.5 with {
            \node[shape=circle, minimum size=3.6pt, fill=col_algo129, inner sep=0.5pt, transform shape] {};
        }
    },
    postaction={decorate}
}}
\tikzstyle{srpath129}=[very thick, col_algo129, >=latex]
\tikzstyle{srpath129_tilfa}=[srpath129, dotted]
\tikzstyle{srpath129_alt}=[srpath129, dashed, col_TOFIX]

\tikzstyle{bigarrow}=[->, line width=5mm, color=gray, >={Latex[width=12mm,length=5mm]}]

\tikzstyle{failed_link}=[starburst, inner sep=0cm, starburst points=11, starburst point height=.17cm, minimum width=.7cm, minimum height=.5cm]
\tikzstyle{failed_router}=[router, starburst, starburst points=11, starburst point height=.3cm, draw=none, inner sep=.05cm]

\tikzset{node distance=2cm}

\tikzstyle{mydomain}=[rectangle, rounded corners, minimum width=3cm, minimum height=1cm,text centered, draw=black, fill=red!30]

\tikzstyle{cpsession}=[very thick, <->, >=latex]
\tikzstyle{cpsession_ebgpv4}=[cpsession, EBGPV4, dotted]
\tikzstyle{cpsession_ibgpv4}=[cpsession, IBGPV4, dotted]
\tikzstyle{cpsession_ebgpv6}=[cpsession, EBGPV6, dashed]
\tikzstyle{cpsession_ibgpv6}=[cpsession, IBGPV6, dashed]

\tikzset{bgpupdate/.style={align=left, draw, very thick, fill=white}}
\tikzset{bgpupdate_ebgpv4/.style={bgpupdate, draw=EBGPV4, dotted}}
\tikzset{bgpupdate_ibgpv4/.style={bgpupdate, draw=IBGPV4, dotted}}
\tikzset{bgpupdate_ebgpv6/.style={bgpupdate, draw=EBGPV6, dashed}}
\tikzset{bgpupdate_ibgpv6/.style={bgpupdate, draw=IBGPV6, dashed}}

%\tikzstyle{cplabel}=[align=center, scale=.6]
%\tikzstyle{rtrlabel}=[align=center, scale=.6]
\tikzstyle{cplabel}=[align=center]
\tikzstyle{rtrlabel}=[align=center]

\tikzstyle{domainnode}=[draw, rectangle, rounded corners=5mm, dashed, very thick]
%\tikzset{mydomain/.style={rectangle, rounded corners, inner sep=1mm, draw=black, fill=white, thick, dashed}}
\tikzstyle{domain0}=[domainnode, DOMZERO]
\tikzstyle{domain1}=[domainnode, DOMONE]
\tikzstyle{domain2}=[domainnode, DOMTWO]

\tikzset{domainVrf/.style={domainnode, rounded corners=2mm, inner sep=2mm}}
\tikzstyle{domainVrfGreen}=[domainVrf, VRFGREEN]
\tikzstyle{domainVrfBlue}=[domainVrf, VRFBLUE, dotted]
\tikzstyle{domainEVI100}=[domainVrf, EVI100]

\tikzset{VRFBLUEPATTERN/.style={pattern={Dots[angle=45,distance={12pt/sqrt(2)}]}, pattern color=VRFBLUE}}

%\tikzset{trianglepost/.style={
%    decoration={
%        markings,
%        mark=between positions 0.1 and 0.9 step 10mm with {
%            \node[shape=regular polygon, regular polygon sides=3, minimum size=4pt, fill=green, inner sep=0.5pt, transform shape, rotate=30] {};
%        }
%    },
%    postaction={decorate}
%}}
%\tikzset{trianglemid/.style={
%    decoration={
%        markings,
%        mark=at position 0.5 with {
%            \node[shape=regular polygon, regular polygon sides=3, minimum size=4pt, fill=green, inner sep=0.5pt, transform shape, rotate=30] {};
%        }
%    },
%    postaction={decorate}
%}}

\fi


\tikzset{computer/.pic={
    \def\compcolor{gray!50}
    % Monitor
    \node[box, rounded corners=0.5, minimum width=12.48, minimum height=9.15, fill=\compcolor, draw=\compcolor] at (0,0)  (rect1){};
    \node at (rect1) [box, minimum width=10.6, minimum height=6.65, fill=white] (rect2){};
    \node at (rect1.south) [box, anchor=north, minimum width=2.5, minimum height=1.3, fill=\compcolor, draw=\compcolor] (rect3){};
    \node at (rect3.south) [box, anchor=north, minimum width=6.27, minimum height=1.3, fill=\compcolor, draw=\compcolor] (rect4){};

    \node[box, rounded corners=0.5, minimum width=5.81, minimum height=9.15+1.3+1.3, fill=\compcolor, draw=\compcolor, anchor=north west, xshift=0.8] at (rect1.north east)  (rect11){};
    \node[box, minimum width=4.12, minimum height=1.27, anchor=north, yshift=-0.8, fill=white, draw=white] at (rect11.north) (rect12){};
    \node[box, minimum width=4.12, minimum height=1.27, anchor=north, yshift=-0.8, fill=white, draw=white] at (rect12.south) (rect13){};
    \node[draw, circle, minimum width=1.2, fill=white, draw=whiteanchor=south, yshift=0.8, transform shape, inner sep=0] at (rect11.south) (rect14){};
    % surrounding box
    \node(-c)[fit=(rect1) (rect2) (rect3) (rect4) (rect11) (rect12) (rect13) (rect14), draw, red, inner sep=0]{};
}}

\tikzset{screen/.pic={
    \def\compcolor{gray!90}
    % Monitor
    \node[box, rounded corners=0.5, minimum width=12.48, minimum height=9.15, fill=\compcolor, \compcolor] at (0,0)(rect1){};
    \node at (rect1) [box, minimum width=10.6, minimum height=6.65, fill=white, font=\tiny, draw=white] (rect2){\tikzpictext};
    \node at (rect1.south) [box, anchor=north, minimum width=2.5, minimum height=1.3, fill=\compcolor, \compcolor] (rect3){};
    \node at (rect3.south) [box, anchor=north, minimum width=6.27, minimum height=1.3, fill=\compcolor, \compcolor] (rect4){};

    % surrounding box
    \node(-c)[rectangle, fit=(rect1) (rect2) (rect3) (rect4), inner sep=0, outer sep=0]{};
}}

  %\pic at (0,4)(comp0) {computer};
  %\node at (2,4)(n1)[draw]{X};
  %\pic at (4,4)(comp1) {computer};

  %\draw (comp0-c) -- (n1);
  %\draw (n1) -- (comp1-c);
  %\node at (comp0-c)[red, draw]{X};



% remove the additional space after ":" when using tt text
% see https://tex.stackexchange.com/questions/118455/why-is-there-extra-space-after-in-texttt
\AddToHook{cmd/ttfamily/after}{\frenchspacing}

\tikzset{%
  show curve controls/.style={
    postaction={
      decoration={
        show path construction,
        curveto code={
          \draw [blue] 
            (\tikzinputsegmentfirst) -- (\tikzinputsegmentsupporta)
            (\tikzinputsegmentlast) -- (\tikzinputsegmentsupportb);
          \fill [red, opacity=0.5] 
            (\tikzinputsegmentsupporta) circle [radius=.2ex]
            (\tikzinputsegmentsupportb) circle [radius=.2ex];
        }
      },
      decorate
}}}

\newcolumntype{L}[1]{>{\raggedright\arraybackslash}m{#1}}
\newcolumntype{C}[1]{>{\centering\arraybackslash}m{#1}}
\newcolumntype{R}[1]{>{\raggedleft\arraybackslash}m{#1}}
%
%\NewEnviron{rndtable}[1]{%
%  \addtolength{\extrarowheight}{1ex}%
%  \rowcolors{2}{tablecolor!20}{tablecolor!40}%
%  \savebox{\tablebox}{%
%    \begin{tabular}{#1}%
%      \BODY%
%    \end{tabular}}%
%  \begin{tikzpicture}
%    \begin{scope}
%      \clip[rounded corners=1ex] (0,-\dp\tablebox) -- (\wd\tablebox,-\dp\tablebox) -- (\wd\tablebox,\ht\tablebox) -- (0,\ht\tablebox) -- cycle;
%      \node at (0,-\dp\tablebox) [anchor=south west,inner sep=0pt]{\usebox{\tablebox}};
%    \end{scope}
%    \draw[rounded corners=1ex] (0,-\dp\tablebox) -- (\wd\tablebox,-\dp\tablebox) -- (\wd\tablebox,\ht\tablebox) -- (0,\ht\tablebox) -- cycle;
%  \end{tikzpicture}
%}

\newlength{\mywidth}
\newlength{\myheight}

% adapted from https://tex.stackexchange.com/questions/616317/tikz-determining-the-size-in-cm-of-a-finished-tikz-picture
\makeatletter
\newcommand{\pgfsize}[2]{ % #1 = width, #2 = height
    \pgfextractx{\@tempdima}{\pgfpointdiff{\pgfpointanchor{current bounding box}{south west}}
        {\pgfpointanchor{current bounding box}{north east}}}
    \global#1=\@tempdima
        \pgfextracty{\@tempdima}{\pgfpointdiff{\pgfpointanchor{current bounding box}{south west}}
            {\pgfpointanchor{current bounding box}{north east}}}
    \global#2=\@tempdima
}
\def\convertto#1#2{\strip@pt\dimexpr #2*65536/\number\dimexpr 1#1}
\makeatother
% add right before \end{tikzpicture}:
% \pgfsize{\mywidth}{\myheight}
% then between \end{tikzpicture} and \end{document} you can use \mywidth and \myheight
% e.g. just log it:
%    \typeout{W=\the\mywidth}
%    \typeout{H=\the\myheight}
% or add it in the doc:
%Width = \convertto{cm}{\the\mywidth} cm -
%Height = \convertto{cm}{\the\myheight} cm
% to manually convert the TeX "pt" value to somethig else, see
% https://tex.stackexchange.com/questions/207162/how-can-i-determine-the-max-dimensions-to-use-for-an-image-im-adding-to-my-docu
% 72.27 TeX pt per Inch, 300 DPI as example
% width_px = floor((345.0 pt)*(1 in / 72.27 pt)*(300 px / 1 in))
% width_px = floor(1432.129514 px)
% width_px = 1432 px


% adjust the line widths
\newlength\mylinewidth
\setlength\mylinewidth{0.6pt}

\tikzset{
    ultra thin/.style= {line width=0.25\mylinewidth},
    very thin/.style=  {line width=0.5\mylinewidth},
    thin/.style=       {line width=\mylinewidth},
    semithick/.style=  {line width=1.5\mylinewidth},
    thick/.style=      {line width=2\mylinewidth},
    very thick/.style= {line width=3\mylinewidth},
    ultra thick/.style={line width=4\mylinewidth},
    every picture/.style={thin}
    every node/.style={thin},
}
% removed path from the above list
%    every path/.style={thin}


